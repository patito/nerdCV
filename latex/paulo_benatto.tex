% LaTeX file for resume 
% This file uses the resume document class (res.cls)

\documentclass[margin]{res}
\usepackage [brazil]{babel}     % nomes e hifenaçã em português
\usepackage{color}
 
\usepackage{t1enc}              % Permite digitar os acentos de forma normal
\usepackage[utf8]{inputenc} 
\usepackage{tikzsymbols}

\topmargin=-0.5in  % start text higher on the page
\setlength{\textheight}{10in} % increase text height to fit resume on 1 page
\begin{document}  
\name{\textit{Paulo Leonardo Benatto}}

\address{Essen, DE \\ benatto@gmail.com \\ Phone: +49 017677556352 \\ Post Code: 45130 }
                           
                        
\begin{resume}
 
\section{Summary}

Nerd, hobbit, dwarf. These are the words I had heard most when I was younger. No, don't get me wrong
I would be happy to be Froddo, however I think I'm more like Smeagol. Anyway if you don't know what
I'm talking about, no worries I'm nerd \Laughey[1.4].
 
Hi, my name is Paulo Leonardo Benatto, I'm a passionate, egoless and eternal learner software developer
from south Brazil. I lived five amazing years in Brighton, United Kingdom, working as Linux Sysadmin and
Software Developer for great companies. Nowadays I’m based in Essen, Germany.

I do believe that I’m friendly and a people person, what does that mean? People as first-class objects :)
(nerds) and I do support the concept of the word “ubuntu”.

Apart from my soft skills, I have a mixed professional experience, I started my life in Europe working as
a busser in a busy restaurant in central London, but as developer most of my experience is around python
ecosystem (Flask, SqlAlchemy, PyTest, Mypy, PEP8 and so on) and I have some golang experience as well.

I also have worked for some years with C ANSI <3, VI editor and Linux environment. I was addicted and
I made use of: VI, Valgrind, GLib, Splint, Double Free and Segfault. Nowadays I use C only to say: "C u later".
 
If you are reading my profile feel free to get in contact or invite me for a pint ;).
 
\section{Education}	Universidade Estadual do Oeste do Parana, BSc in Computer Science, December 2007.

\section{Experience}


\vspace{-0.1in}
   \begin{tabbing}
   \hspace{2.3in}\= \hspace{1.7in}\= \kill
    \textbf{Naontek}    \>\>\textbf{Jan 2020 - Aug 2020}\\
    \textit{Software Developer}\\        
    \textbf{Main Technologies}: Python, Go, AWS and Git;
   \end{tabbing}\vspace{-20pt}
    \vspace{2mm}
I worked as a backend developer building a platform called univiva. Univiva is the platform through which
health care professionals can find and use third-party services in a variety of topics relating to their
practice and pharmacy in a targeted, transparent, independent and convenient way.

The stack used to develop the platform was:
    \begin{itemize}
		    \item \textbf{Programming Languages}: Golang, Python and JavaScript 
		    \item \textbf{Continuous Methodologies}: Gitlab CI/CD    
		    \item \textbf{AWS}: Lambda Functions, SNS, S3, RDS
            \item \textbf{Container}: Docker and Docker Composer   
    \end{itemize}

\vspace{-0.1in}
   \begin{tabbing}
   \hspace{2.3in}\= \hspace{1.7in}\= \kill
    \textbf{Altagram}    \>\>\textbf{Sep 2019 - Dec 2019}\\
    \textit{Python Developer}\\        
    \textbf{Main Technologies}: Python, Postgres and Git;
   \end{tabbing}\vspace{-20pt}
    \vspace{2mm}
I was part of a team responsible to develop a localization platform for gamming industry.
I worked developing services in Python and deploying to Digital Ocean. For the services the team
made use of Flask Framework (PyTest, PEP8,), Postgres database and Docker.

The stack used to develop the platform was:
    \begin{itemize}
		    \item \textbf{Programming Languages}: Python3
            \item \textbf{Frameworks and tools}: Flask, PyTest, PEP8, 
		    \item \textbf{Continuous Methodologies}: Gitlab CI/CD    
		    \item \textbf{Cloud Environment}: Digital Ocean
            \item \textbf{Container}: Docker and Docker Composer   
    \end{itemize}


\vspace{-0.1in}
   \begin{tabbing}
   \hspace{2.3in}\= \hspace{1.7in}\= \kill
    \textbf{Futrli}    \>\>\textbf{Nov 2018 - Aug 2019}\\
    \textit{Python Developer}\\        
    \textbf{Main Technologies}: Python, AWS and Git;
   \end{tabbing}\vspace{-20pt}
    \vspace{2mm}
I was part of a team responsible to develop products to get a clear, live, daily updating view of your predicted cash,
profit, big payments ahead, hidden trends and areas of risk from analysis of every transaction and every account,
so you’re always in control of the next decision you need to make.   
I worked developing micro services in Python and deploying to AWS (Lambda, S3, Step Functions).

\vspace{-0.1in}
   \begin{tabbing}
   \hspace{2.3in}\= \hspace{1.7in}\= \kill
    \textbf{Brandwatch}    \>\>\textbf{Jul 2016 - Nov 2018}\\
    \textit{Core Systems Engineer}\\        
    \textbf{Main Technologies}: Golang, Python and Git;
   \end{tabbing}\vspace{-20pt}
    \vspace{2mm}
I was part of a team that develops a project called Bigode that is a centralized configuration state
aggregator that presents information on all the significant assets in your IT environment. It is
part of my day developing REST API in Go, CLI tools in python and some contact with JavaScript to develop the frontend.
I also have been automation tasks in python, writing checks for sensu (monitoring) and deploying using puppet.
I worked few months as JavaScript developer maintaing a couple of service in NodeJS.

\vspace{-0.1in}
   \begin{tabbing}
   \hspace{2.3in}\= \hspace{1.7in}\= \kill
    \textbf{Sainsburys}    \>\>\textbf{Feb 2016 - Jul 2016}\\
    \textit{Go Developer}\\        
    \textbf{Main Technologies}: Golang, Git;
   \end{tabbing}\vspace{-20pt}
    \vspace{2mm}
Member of ecommerce service team responsible to develop microservices in Go language and postgres.

\vspace{-0.1in}
   \begin{tabbing}
   \hspace{2.3in}\= \hspace{1.7in}\= \kill % set up two tab positions
    \textbf{Brandwatch}    \>\>\textbf{Jul 2014 - Feb 2016}\\
    \textit{Junior Linux Systems Administrator}\\        
    \textbf{Main Technologies}: Linux, Python, Puppet, Bareos, Ansible, Git;
   \end{tabbing}\vspace{-20pt}      % suppress blank line after tabbing
    \vspace{2mm}
My main task was to keep all Debian GNU/Linux servers, physical and virtual, running as smoothly as possible.
I enjoy automating tasks using ansible and sometimes I code in Python to make my life easier.
My spare time, I like to spend keeping myself updated and studying about programming languages such as Python, Golang and C.


\vspace{-0.1in}
   \begin{tabbing}
   \hspace{2.3in}\= \hspace{1.7in}\= \kill % set up two tab positions
    \textbf{DBA}    \>\>\textbf{Dec 2013 - Feb 2014}\\
    \textit{Software Engineer - Contract}\\        
    \textbf{Main Technologies}: Linux, Python, Raspberry PI and C (GCC, Valgrind, Splint);
   \end{tabbing}\vspace{-20pt}      % suppress blank line after tabbing
    \vspace{2mm}
I was part of the team responsible to develop a platform (hardware and software) to analyse vehicle
traffic on Brazilian highways. My main taks was develop C ANSI library to support the team needs on Linux
environment.

\vspace{-0.1in}
   \begin{tabbing}
   \hspace{2.3in}\= \hspace{1.7in}\= \kill % set up two tab positions
    \textbf{SEC+}    \>\>\textbf{Dec 2012 - Dec 2013}\\
    \textit{Software Engineer}\\        
    \textbf{Main Technologies}: Linux, Python, Django Framework, JavaScript and C;
   \end{tabbing}\vspace{-20pt}      % suppress blank line after tabbing
    \vspace{2mm}
     I worked with backend development of web system for intelligent monitoring 
     and management of natural disasters using Python and the Django framework. 
     Front-end with Javascript (JQuery, Bootstrap, Google Maps API),
     JSON, HTML5, CSS.

   \begin{tabbing}
   \hspace{2.3in}\= \hspace{1.7in}\= \kill % set up two tab positions
    \textbf{Digitro Technology - NDS}    \>\>\textbf{Jan 2012 - Dec 2012}\\
    \textit{Software Engineer}\\   
    \textbf{Main Technologies}: Linux, C/C++, Wireshark, ShellScript;
   \end{tabbing}\vspace{-20pt}      % suppress blank line after tabbing
    \vspace{2mm}
    I was a member of the team responsible to design and develop the roadmap features of a product called
    "Guardião" used to interecept calls and internet traffic (Facebook, WhatsApp and Hotmail) of people who are 
    being investigated by the police.

   %\vspace{1mm}
   \begin{tabbing}
   \hspace{2.3in}\= \hspace{1.5in}\= \kill % set up two tab positions
    \textbf{Digitro Technology - STE}    \>\>\textbf{Set 2008 - Dec 2011}\\
    \textit{Software Engineer}\\   
    \textbf{Main Technologies}: Linux, VoIP, C/C++, shellscript, protocols: UDP, TCP, SIP;
   \end{tabbing}\vspace{-20pt}      % suppress blank line after tabbing
    \vspace{2mm}
    
    I worked on the design and development of VoIP products such as PBX, softphone and IP phone. 
    All projects were developed using C language in Linux and Windows environment.


\vspace{10mm}     

\section{Open Source Projects}
		\begin{itemize}
		    \vspace{2mm}
		    \item \textbf{libpenetra}: The libpenetra was created with the goal of studying the windows binary format 
		                               known as Portable Executable (PE). With libpenetra you can access all information
		                               about PE binaries. (\texttt{https://github.com/patito/libpenetra}) \vspace{1mm}
		                               
		    \item \textbf{libmalelf}: The libmalelf is an evil library that SHOULD be used for good! It was developed
		                              with the intent to assist in the process of infecting binaries and provide a safe 
		                              way to analyze malwares. (\texttt{https://github.com/SecPlus/libmalelf})\vspace{1mm}
		                              
		    \item \textbf{malelf}: Malelf is a tool that uses libmalelf to dissect and infect ELF binaries. 
		                           (\texttt{https://github.com/SecPlus/malelf})
		\end{itemize}
 
\section{More Info}
    \begin{itemize}
        \item \textbf{Linkedin}: http://www.linkedin.com/in/benatto
        \item \textbf{Github}: https://github.com/patito
        \item \textbf{Gitlab}: https://gitlab.com/benatto
        \item \textbf{Blog}: http://patito.github.io
    \end{itemize}


\end{resume} 
\end{document}









